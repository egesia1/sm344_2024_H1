\documentclass{article}
\usepackage{amsmath}
\usepackage{graphicx}

\begin{document}

\title{Rizwan Homework }
\author{}
\date{}
\maketitle

\section*{Exercise 4.14}

Using the \textit{Students} data file, for the corresponding population, construct a 95\% confidence interval:
\begin{enumerate}
    \item[(a)] for the mean weekly number of hours spent watching TV.
    \item[(b)] to compare females and males on the mean weekly number of hours spent watching TV.
\end{enumerate}

In each case, state assumptions, including the practical importance of each, and interpret results.

\section*{Solution}

\subsection*{(a) 95\% Confidence Interval for Mean Weekly Hours of TV Watching}

Let:
\[
\bar{X} = \text{sample mean of weekly hours spent watching TV},
\]
\[
s = \text{sample standard deviation of weekly hours},
\]
\[
n = \text{sample size}.
\]

The standard error of the mean is:
\[
\text{SE} = \frac{s}{\sqrt{n}}.
\]

The 95\% confidence interval is:
\[
\bar{X} \pm t_{\alpha/2, \, n-1} \times \text{SE}
\]
where \( t_{\alpha/2, \, n-1} \) is the critical value from the \( t \)-distribution with \( n-1 \) degrees of freedom.

\textbf{Assumptions:} The data is randomly sampled, and either normally distributed or sufficiently large for the Central Limit Theorem to apply.

\textbf{Interpretation:} This interval provides a range within which the true mean weekly hours of TV watching is expected to fall with 95\% confidence.

\subsection*{(b) 95\% Confidence Interval for the Difference in Mean Weekly Hours Between Females and Males}

Let:
\[
\bar{X}_f = \text{mean weekly hours for females}, \quad s_f = \text{standard deviation for females},
\]
\[
\bar{X}_m = \text{mean weekly hours for males}, \quad s_m = \text{standard deviation for males},
\]
\[
n_f = \text{sample size for females}, \quad n_m = \text{sample size for males}.
\]

The pooled standard error is:
\[
\text{SE}_{\text{pooled}} = \sqrt{\frac{s_f^2}{n_f} + \frac{s_m^2}{n_m}}.
\]

The 95\% confidence interval for the difference in means (\( \mu_f - \mu_m \)) is:
\[
(\bar{X}_f - \bar{X}_m) \pm t_{\alpha/2, \, df} \times \text{SE}_{\text{pooled}}
\]
where \( t_{\alpha/2, \, df} \) is the critical \( t \)-value with degrees of freedom \( df \) (calculated as \( \min(n_f - 1, n_m - 1) \) or using the Satterthwaite approximation if variances are unequal).

\textbf{Assumptions:} Samples are independent and randomly selected. We assume normality or large sample sizes, and equal variances for females and males (if not, use Welch’s \( t \)-test).

\textbf{Interpretation:} This interval gives a range for the true difference in mean weekly hours between females and males. If the interval includes 0, it suggests no significant difference in weekly hours spent watching TV by gender.

\section*{Exercise 4.16}

The \textit{Substance} data file shows a contingency table formed from a survey that asked a sample of high school students whether they have ever used alcohol, cigarettes, and marijuana. Construct a 95\% Wald confidence interval to compare those who have used or not used alcohol on whether they have used marijuana, using:
\begin{enumerate}
    \item[(a)] formula (4.13);
    \item[(b)] software.
\end{enumerate}

State assumptions for your analysis, and interpret results.

\section*{Solution}

\subsection*{Contingency Table Information}

Let:
\begin{itemize}
    \item \( n_{11} \) = number of students who have used both alcohol and marijuana,
    \item \( n_{10} \) = number of students who have used alcohol but not marijuana,
    \item \( n_{01} \) = number of students who have not used alcohol but have used marijuana,
    \item \( n_{00} \) = number of students who have not used either alcohol or marijuana.
\end{itemize}

Define:
\[
n_1 = n_{11} + n_{10} \quad \text{(total who have used alcohol)}
\]
\[
n_0 = n_{01} + n_{00} \quad \text{(total who have not used alcohol)}
\]
\[
p_1 = \frac{n_{11}}{n_1} \quad \text{(proportion of alcohol users who have used marijuana)}
\]
\[
p_0 = \frac{n_{01}}{n_0} \quad \text{(proportion of non-alcohol users who have used marijuana)}
\]

The difference in proportions is:
\[
\hat{p} = p_1 - p_0.
\]

\subsection*{(a) 95\% Wald Confidence Interval using Formula (4.13)}

The formula for the 95\% Wald confidence interval for the difference between two proportions is given by:
\[
\hat{p} \pm z_{\alpha/2} \sqrt{\frac{p_1(1 - p_1)}{n_1} + \frac{p_0(1 - p_0)}{n_0}}
\]
where \( z_{\alpha/2} \) is the critical value from the standard normal distribution for a 95\% confidence level (approximately 1.96).

Substitute \( p_1 \), \( p_0 \), \( n_1 \), and \( n_0 \) from the data to calculate the interval.

\textbf{Assumptions:}
\begin{itemize}
    \item Random sampling: The sample of high school students should be representative of the population.
    \item Independence: The responses of individual students are independent of each other.
    \item Normal approximation: Sample sizes \( n_1 \) and \( n_0 \) should be large enough for the normal approximation to hold (usually \( n \times p \geq 5 \) and \( n \times (1 - p) \geq 5 \) for each group).
\end{itemize}

\textbf{Interpretation:} The confidence interval provides a range for the difference in proportions between alcohol users and non-users in terms of marijuana usage. If the interval includes 0, it suggests no significant difference in marijuana use between those who have and have not used alcohol.

\subsection*{(b) 95\% Confidence Interval using Software}

To compute this confidence interval using software as R, calculating confidence intervals for two proportions. For example, in R.

Code in R:
\begin{verbatim}
# Suppose the counts are:
# n11 = 50, n10 = 200, n01 = 30, n00 = 220

n1 <- 50 + 200
n0 <- 30 + 220
p1 <- 50 / n1
p0 <- 30 / n0
prop.test(x = c(50, 30), n = c(n1, n0), conf.level = 0.95)
\end{verbatim}

This will return the confidence interval for the difference in proportions \( p_1 - p_0 \).

\textbf{Interpretation:} The software output provides the same interpretation as in part (a), with potentially more precise results based on the exact data and computational methods of the software.

\section*{Exercise 4.48}

For a simple random sample of \( n \) subjects, explain why it is about 95\% likely that the sample proportion has error no more than \( \frac{1}{\sqrt{n}} \) in estimating the population proportion. (Hint: To show this "\( \frac{1}{\sqrt{n}} \)" rule, find two standard errors when \( \pi = 0.50 \), and explain how this compares to two standard errors at other values of \( \pi \).) Using this result, show that \( n = \frac{1}{M^2} \) is a safe sample size for estimating a proportion to within \( M \) with 95\% confidence.

\section*{Solution}

Let:
\[
\hat{p} = \text{sample proportion},
\]
\[
\pi = \text{population proportion}.
\]

The standard error of the sample proportion \( \hat{p} \) is given by:
\[
\text{SE}(\hat{p}) = \sqrt{\frac{\pi (1 - \pi)}{n}}.
\]

To construct a 95\% confidence interval, we use approximately two standard errors (since 95\% of the standard normal distribution falls within \(\pm 1.96\) standard deviations of the mean). Therefore, the margin of error \( E \) at the 95\% confidence level is:
\[
E \approx 2 \times \text{SE}(\hat{p}) = 2 \sqrt{\frac{\pi (1 - \pi)}{n}}.
\]

\subsection*{The \( \frac{1}{\sqrt{n}} \) Rule}

To illustrate the \( \frac{1}{\sqrt{n}} \) rule, consider the case where \( \pi = 0.50 \), which maximizes the product \( \pi(1 - \pi) \) and hence gives the largest possible standard error.

When \( \pi = 0.50 \), the standard error is:
\[
\text{SE}(\hat{p}) = \sqrt{\frac{0.5 \times 0.5}{n}} = \frac{1}{2\sqrt{n}}.
\]

Thus, the margin of error at the 95\% confidence level becomes:
\[
E \approx 2 \times \frac{1}{2\sqrt{n}} = \frac{1}{\sqrt{n}}.
\]

This shows that, for a population proportion near \( 0.5 \), it is about 95\% likely that the sample proportion \( \hat{p} \) will be within \( \frac{1}{\sqrt{n}} \) of the population proportion \( \pi \). For other values of \( \pi \), the standard error \( \sqrt{\pi(1 - \pi)/n} \) will be smaller, making \( \frac{1}{\sqrt{n}} \) a conservative, safe upper bound on the margin of error.

\subsection*{Deriving the Sample Size Formula \( n = \frac{1}{M^2} \)}

To ensure the margin of error \( E \leq M \) with 95\% confidence, we set the margin of error to \( M \):
\[
\frac{1}{\sqrt{n}} \leq M.
\]

Solving for \( n \), we get:
\[
\sqrt{n} \geq \frac{1}{M},
\]
\[
n \geq \frac{1}{M^2}.
\]

Thus, \( n = \frac{1}{M^2} \) is a safe sample size for estimating the population proportion within a margin of error \( M \) with 95\% confidence.

\textbf{Conclusion:} This result provides a guideline for determining sample size when estimating a population proportion. For a desired margin of error \( M \) at the 95\% confidence level, a sample size of \( n = \frac{1}{M^2} \) is sufficient to ensure that the sample proportion \( \hat{p} \) will likely be within \( M \) of the population proportion \( \pi \).
\end{document}
